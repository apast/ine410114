\documentclass[a4paper,oneside]{article}
\usepackage[top=3cm,bottom=2cm,left=3cm,right=2cm]{geometry}
\usepackage[utf8]{inputenc}
\usepackage[brazil]{babel}
%--\usepackage{abntcite}
\usepackage{indentfirst}
%--\usepackage{fancyhdr}

\usepackage[T1]{fontenc}
%--\usepackage{graphicx}
\usepackage{amsmath}
%--\usepackage{url}

%--\usepackage{epstopdf}
%--\usepackage{fixltx2e}
%--\usepackage{makeidx}

%--\cfoot{}
%--\rfoot{\thepage}

\fontsize{12pt}{1.5cm}

%--\makeindex

\newcommand{\tuple}[1]{\ensuremath{\left \langle #1 \right \rangle }}

\title{Near Field Communication (NFC)}
\author{
    André Pastore Gomes de Santana\\
    \texttt{andrepgs@inf.ufsc.br}
    \and
    Tarcísio Eduardo Moreira Crocomo
    \texttt{tarcisio.crocomo@inf.ufsc.br}
}

\date{\today}

\begin{document}
\maketitle


\section{Introdução}
Comunicação por Campo Próximo (Near Field Communication, NFC) define uma tecnologia de comunicação sem fio entre dois dispositivos por proximidade. Esta tecnologia foi construida a partir da especificação de Radio Frequency Identification (RFID)\cite{iso14443-2}.

\section{Visão Geral da Pilha de Protocolos}
\section{Camada Física}


\section{Camada de Enlace}
\section{Camada de Rede no que concerne suporte a IP}




\bibliographystyle{plain}
\bibliography{resources/bib/pastore}
\end{document}
